% arara: pdflatex
% arara: pdflatex
% this file is an adapted version of acrotest.tex shipped out with the `acronym'
% package
\documentclass{article}
\usepackage[T1]{fontenc}
\usepackage[utf8]{inputenc}
\usepackage[english]{babel}
\usepackage[colorlinks]{hyperref}
\usepackage{acro}

\acsetup{
  sort = true ,
  page-style = comma ,
  extra-style = paren ,
  hyperref =true
}

\DeclareAcronym{CDMA}{
  short            = CDMA ,
  long             = Code Division Multiple Access ,
  long-plural      = es
}
\DeclareAcronym{GSM}{
  short            = GSM ,
  long             = Global System for Mobile communication
}
\DeclareAcronym{NA}{
  short            = {\ensuremath{N_{\mathrm{A}}}} ,
  long             = Number of Avogadro ,
  extra            = see \S\ref{Chem} ,
  pdfstring        = NA
}
\DeclareAcronym{NAD+}{
  short            = {NAD\textsuperscript{+}} ,
  short-plural     = ,
  long             = Nicotinamide Adenine Dinucleotide ,
  pdfstring        = NAD+
}
\DeclareAcronym{NUA}{
  short            = NUA ,
  long             = Not Used Acronym
}
\DeclareAcronym{TDMA}{
  short            = TDMA ,
  long             = Time Division Multiple Access ,
  long-plural      = es
}
\DeclareAcronym{UA}{
  short            = UA ,
  long             = Used Acronym
}
\DeclareAcronym{lox}{
  short            = {\emph{LOX}} ,
  long             = Liquid Oxygen ,
  pdfstring        = LOX
}
\DeclareAcronym{lh2}{
  short            = {\emph{LH\textsubscript{2}}} ,
  long             = Liquid Hydrogen ,
  pdfstring        = LH2
}
\DeclareAcronym{IC}{
  short            = IC ,
  long             = Integrated Circuit
}
\DeclareAcronym{BUT}{
  short            = BUT ,
  long             = Block Under Test ,
  long-plural-form = Blocks Under Test
}
\begin{document}

\section{Intro}
In the early nineties, \acs{GSM} was deployed in many European
countries. \ac{GSM} offered for the first time international
roaming for mobile subscribers. The \acs{GSM}'s use of \ac{TDMA} as
its communication standard was debated at length. And every now
and then there are big discussion whether \ac{CDMA} should have
been chosen over \ac{TDMA}.

\section{Furthermore}
\acresetall
The reader could have forgotten all the nice acronyms, so we repeat the
meaning again.

If you want to know more about \acf{GSM}, \acf{TDMA}, \acf{CDMA}
and other acronyms, just read a book about mobile communication. Just
to mention it: There is another \ac{UA}, just for testing purposes!

\begin{figure}[h]
Figure
\caption{A float also admits references like \ac{GSM} or \acf{CDMA}.}
\end{figure}

\subsection{Some chemistry and physics}
\label{Chem}
\ac{NAD+} is a major electron acceptor in the oxidation
of fuel molecules. The reactive part of \ac{NAD+} is its nictinamide
ring, a pyridine derivate.

One mol consists of \acs{NA} atoms or molecules. There is a relation
between the constant of Boltzmann and the \acl{NA}:
\begin{equation}
  k = R/\acs{NA}
\end{equation}

\acl{lox}/\acl{lh2} (\acs{lox}/\acs{lh2})

\subsection{Some testing fundamentals}
When testing \acp{IC}, one typically wants to identify functional
blocks to be tested separately. The latter are commonly indicated as
\acp{BUT}. To test a \ac{BUT} requires defining a testing strategy\dots

\printacronyms

\end{document}
